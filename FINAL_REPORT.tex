\documentclass[]{article}
\usepackage{lmodern}
\usepackage{amssymb,amsmath}
\usepackage{ifxetex,ifluatex}
\usepackage{fixltx2e} % provides \textsubscript
\ifnum 0\ifxetex 1\fi\ifluatex 1\fi=0 % if pdftex
  \usepackage[T1]{fontenc}
  \usepackage[utf8]{inputenc}
\else % if luatex or xelatex
  \ifxetex
    \usepackage{mathspec}
  \else
    \usepackage{fontspec}
  \fi
  \defaultfontfeatures{Ligatures=TeX,Scale=MatchLowercase}
\fi
% use upquote if available, for straight quotes in verbatim environments
\IfFileExists{upquote.sty}{\usepackage{upquote}}{}
% use microtype if available
\IfFileExists{microtype.sty}{%
\usepackage{microtype}
\UseMicrotypeSet[protrusion]{basicmath} % disable protrusion for tt fonts
}{}
\usepackage[margin=1in]{geometry}
\usepackage{hyperref}
\hypersetup{unicode=true,
            pdftitle={Final Report},
            pdfauthor={Kecheng Liang},
            pdfborder={0 0 0},
            breaklinks=true}
\urlstyle{same}  % don't use monospace font for urls
\usepackage{color}
\usepackage{fancyvrb}
\newcommand{\VerbBar}{|}
\newcommand{\VERB}{\Verb[commandchars=\\\{\}]}
\DefineVerbatimEnvironment{Highlighting}{Verbatim}{commandchars=\\\{\}}
% Add ',fontsize=\small' for more characters per line
\usepackage{framed}
\definecolor{shadecolor}{RGB}{248,248,248}
\newenvironment{Shaded}{\begin{snugshade}}{\end{snugshade}}
\newcommand{\KeywordTok}[1]{\textcolor[rgb]{0.13,0.29,0.53}{\textbf{#1}}}
\newcommand{\DataTypeTok}[1]{\textcolor[rgb]{0.13,0.29,0.53}{#1}}
\newcommand{\DecValTok}[1]{\textcolor[rgb]{0.00,0.00,0.81}{#1}}
\newcommand{\BaseNTok}[1]{\textcolor[rgb]{0.00,0.00,0.81}{#1}}
\newcommand{\FloatTok}[1]{\textcolor[rgb]{0.00,0.00,0.81}{#1}}
\newcommand{\ConstantTok}[1]{\textcolor[rgb]{0.00,0.00,0.00}{#1}}
\newcommand{\CharTok}[1]{\textcolor[rgb]{0.31,0.60,0.02}{#1}}
\newcommand{\SpecialCharTok}[1]{\textcolor[rgb]{0.00,0.00,0.00}{#1}}
\newcommand{\StringTok}[1]{\textcolor[rgb]{0.31,0.60,0.02}{#1}}
\newcommand{\VerbatimStringTok}[1]{\textcolor[rgb]{0.31,0.60,0.02}{#1}}
\newcommand{\SpecialStringTok}[1]{\textcolor[rgb]{0.31,0.60,0.02}{#1}}
\newcommand{\ImportTok}[1]{#1}
\newcommand{\CommentTok}[1]{\textcolor[rgb]{0.56,0.35,0.01}{\textit{#1}}}
\newcommand{\DocumentationTok}[1]{\textcolor[rgb]{0.56,0.35,0.01}{\textbf{\textit{#1}}}}
\newcommand{\AnnotationTok}[1]{\textcolor[rgb]{0.56,0.35,0.01}{\textbf{\textit{#1}}}}
\newcommand{\CommentVarTok}[1]{\textcolor[rgb]{0.56,0.35,0.01}{\textbf{\textit{#1}}}}
\newcommand{\OtherTok}[1]{\textcolor[rgb]{0.56,0.35,0.01}{#1}}
\newcommand{\FunctionTok}[1]{\textcolor[rgb]{0.00,0.00,0.00}{#1}}
\newcommand{\VariableTok}[1]{\textcolor[rgb]{0.00,0.00,0.00}{#1}}
\newcommand{\ControlFlowTok}[1]{\textcolor[rgb]{0.13,0.29,0.53}{\textbf{#1}}}
\newcommand{\OperatorTok}[1]{\textcolor[rgb]{0.81,0.36,0.00}{\textbf{#1}}}
\newcommand{\BuiltInTok}[1]{#1}
\newcommand{\ExtensionTok}[1]{#1}
\newcommand{\PreprocessorTok}[1]{\textcolor[rgb]{0.56,0.35,0.01}{\textit{#1}}}
\newcommand{\AttributeTok}[1]{\textcolor[rgb]{0.77,0.63,0.00}{#1}}
\newcommand{\RegionMarkerTok}[1]{#1}
\newcommand{\InformationTok}[1]{\textcolor[rgb]{0.56,0.35,0.01}{\textbf{\textit{#1}}}}
\newcommand{\WarningTok}[1]{\textcolor[rgb]{0.56,0.35,0.01}{\textbf{\textit{#1}}}}
\newcommand{\AlertTok}[1]{\textcolor[rgb]{0.94,0.16,0.16}{#1}}
\newcommand{\ErrorTok}[1]{\textcolor[rgb]{0.64,0.00,0.00}{\textbf{#1}}}
\newcommand{\NormalTok}[1]{#1}
\usepackage{graphicx,grffile}
\makeatletter
\def\maxwidth{\ifdim\Gin@nat@width>\linewidth\linewidth\else\Gin@nat@width\fi}
\def\maxheight{\ifdim\Gin@nat@height>\textheight\textheight\else\Gin@nat@height\fi}
\makeatother
% Scale images if necessary, so that they will not overflow the page
% margins by default, and it is still possible to overwrite the defaults
% using explicit options in \includegraphics[width, height, ...]{}
\setkeys{Gin}{width=\maxwidth,height=\maxheight,keepaspectratio}
\IfFileExists{parskip.sty}{%
\usepackage{parskip}
}{% else
\setlength{\parindent}{0pt}
\setlength{\parskip}{6pt plus 2pt minus 1pt}
}
\setlength{\emergencystretch}{3em}  % prevent overfull lines
\providecommand{\tightlist}{%
  \setlength{\itemsep}{0pt}\setlength{\parskip}{0pt}}
\setcounter{secnumdepth}{0}
% Redefines (sub)paragraphs to behave more like sections
\ifx\paragraph\undefined\else
\let\oldparagraph\paragraph
\renewcommand{\paragraph}[1]{\oldparagraph{#1}\mbox{}}
\fi
\ifx\subparagraph\undefined\else
\let\oldsubparagraph\subparagraph
\renewcommand{\subparagraph}[1]{\oldsubparagraph{#1}\mbox{}}
\fi

%%% Use protect on footnotes to avoid problems with footnotes in titles
\let\rmarkdownfootnote\footnote%
\def\footnote{\protect\rmarkdownfootnote}

%%% Change title format to be more compact
\usepackage{titling}

% Create subtitle command for use in maketitle
\newcommand{\subtitle}[1]{
  \posttitle{
    \begin{center}\large#1\end{center}
    }
}

\setlength{\droptitle}{-2em}

  \title{Final Report}
    \pretitle{\vspace{\droptitle}\centering\huge}
  \posttitle{\par}
    \author{Kecheng Liang}
    \preauthor{\centering\large\emph}
  \postauthor{\par}
      \predate{\centering\large\emph}
  \postdate{\par}
    \date{12/12/2018}


\begin{document}
\maketitle

\section{Introduction}\label{introduction}

Basketball is one of the most popular sports in the world. National
Basketball Association(NBA) is the largest league for this sport. There
are lots of interesting data in the game. In basketball, an assist is
attributed to a player who passes the ball to a teammate in a way that
leads to a score by field goal, meaning that he or she was ``assisting''
in the basket. Because an assist can be scored for the passer even if
the player who receives the pass makes a basket after dribbling the
ball. In some situtation it becomes hard to define whether it is a
assist. We may think that player who play in the home game are more
easily get the tenth assist when the player already have nine assists.
Same thing may happen in rebound. I want to do the analysis whether it
really happens. Also I will do other interesting graph to show the
miracle NBA data.

\section{Data}\label{data}

The data is downloaded from website and the data from 2012 to 2018. It
is well organized with 51 variables and I removed some useless
variables.

\begin{Shaded}
\begin{Highlighting}[]
\NormalTok{Initial <-}\StringTok{ }\KeywordTok{read.csv}\NormalTok{(}\StringTok{"2012-18_playerBoxScore.csv"}\NormalTok{)}
\NormalTok{Initial}\OperatorTok{$}\NormalTok{playAST <-}\StringTok{ }\KeywordTok{as.numeric}\NormalTok{(Initial}\OperatorTok{$}\NormalTok{playAST)}
\NormalTok{Initial}\OperatorTok{$}\NormalTok{playTRB <-}\StringTok{ }\KeywordTok{as.numeric}\NormalTok{(}\KeywordTok{as.character}\NormalTok{(Initial}\OperatorTok{$}\NormalTok{playTRB))}
\end{Highlighting}
\end{Shaded}

\begin{verbatim}
## Warning: 强制改变过程中产生了NA
\end{verbatim}

\begin{Shaded}
\begin{Highlighting}[]
\NormalTok{Initial <-}\StringTok{ }\NormalTok{Initial[,}\KeywordTok{c}\NormalTok{(}\OperatorTok{-}\DecValTok{1}\NormalTok{,}\OperatorTok{-}\DecValTok{2}\NormalTok{,}\OperatorTok{-}\DecValTok{3}\NormalTok{,}\OperatorTok{-}\DecValTok{4}\NormalTok{,}\OperatorTok{-}\DecValTok{5}\NormalTok{,}\OperatorTok{-}\DecValTok{8}\NormalTok{,}\OperatorTok{-}\DecValTok{11}\NormalTok{,}\OperatorTok{-}\DecValTok{12}\NormalTok{,}\OperatorTok{-}\DecValTok{13}\NormalTok{,}\OperatorTok{-}\DecValTok{14}\NormalTok{,}\OperatorTok{-}\DecValTok{15}\NormalTok{,}\OperatorTok{-}\DecValTok{16}\NormalTok{,}\OperatorTok{-}\DecValTok{17}\NormalTok{,}\OperatorTok{-}\DecValTok{24}\NormalTok{,}\OperatorTok{-}\DecValTok{48}\NormalTok{,}\OperatorTok{-}\DecValTok{49}\NormalTok{,}\OperatorTok{-}\DecValTok{50}\NormalTok{,}\OperatorTok{-}\DecValTok{51}\NormalTok{)]}
\NormalTok{away_data <-}\StringTok{ }\KeywordTok{filter}\NormalTok{(Initial,Initial}\OperatorTok{$}\NormalTok{teamLoc}\OperatorTok{==}\StringTok{"Away"}\NormalTok{)}
\NormalTok{home_data <-}\StringTok{ }\KeywordTok{filter}\NormalTok{(Initial,Initial}\OperatorTok{$}\NormalTok{teamLoc}\OperatorTok{==}\StringTok{"Home"}\NormalTok{)}
\NormalTok{nba_total <-}\StringTok{ }\KeywordTok{rbind}\NormalTok{(away_data,home_data)}
\end{Highlighting}
\end{Shaded}

\section{EDA}\label{eda}

\begin{Shaded}
\begin{Highlighting}[]
\KeywordTok{hist}\NormalTok{(}\DataTypeTok{x=}\NormalTok{away_data}\OperatorTok{$}\NormalTok{playAST,}\DataTypeTok{main =} \StringTok{"Away Assist"}\NormalTok{,}\DataTypeTok{xlab =} \StringTok{"number of assist"}\NormalTok{)}
\KeywordTok{hist}\NormalTok{(}\DataTypeTok{x=}\NormalTok{home_data}\OperatorTok{$}\NormalTok{playAST,}\DataTypeTok{main =} \StringTok{"Home Assist"}\NormalTok{,}\DataTypeTok{xlab =} \StringTok{"number of assist"}\NormalTok{)}
\KeywordTok{hist}\NormalTok{(}\DataTypeTok{x=}\NormalTok{away_data}\OperatorTok{$}\NormalTok{playTRB,}\DataTypeTok{main =} \StringTok{"Away Rebound"}\NormalTok{,}\DataTypeTok{xlab =} \StringTok{"number of rebound"}\NormalTok{)}
\KeywordTok{hist}\NormalTok{(}\DataTypeTok{x=}\NormalTok{home_data}\OperatorTok{$}\NormalTok{playTRB,}\DataTypeTok{main =} \StringTok{"Home Rebound"}\NormalTok{,}\DataTypeTok{xlab =} \StringTok{"number of rebound"}\NormalTok{)}
\end{Highlighting}
\end{Shaded}

\begin{center}\includegraphics{FINAL_REPORT_files/figure-latex/unnamed-chunk-2-1} \includegraphics{FINAL_REPORT_files/figure-latex/unnamed-chunk-2-2} \includegraphics{FINAL_REPORT_files/figure-latex/unnamed-chunk-2-3} \includegraphics{FINAL_REPORT_files/figure-latex/unnamed-chunk-2-4} \end{center}

\section{Chi square test}\label{chi-square-test}

\begin{verbatim}
## 
##  Pearson's Chi-squared test
## 
## data:  total_ast
## X-squared = 95.549, df = 23, p-value = 8.177e-11
\end{verbatim}

\begin{verbatim}
## 
##  Pearson's Chi-squared test
## 
## data:  total_trb
## X-squared = 64.517, df = 29, p-value = 0.000164
\end{verbatim}


\end{document}
